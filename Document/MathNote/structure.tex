%%%%%%%%%%%%%%%%%%%%%%%%%%%%%%%%%%%%%%%%%
% Structural Definitions File
% Version 2.0 (9/2/15)
%
% Original author:
% Mathias Legrand (legrand.mathias@gmail.com) with modifications by:
% Vel (vel@latextemplates.com)
% 
% This file has been downloaded from:
% http://www.LaTeXTemplates.com
%
% License:
% CC BY-NC-SA 3.0 (http://creativecommons.org/licenses/by-nc-sa/3.0/)
%
%%%%%%%%%%%%%%%%%%%%%%%%%%%%%%%%%%%%%%%%%




%----------------------------------------------------------------------------------------
%	
%----------------------------------------------------------------------------------------

\providecommand{\varCIECode}{}
\providecommand{\varCIEVar}{}

\providecommand{\iPageStart}{1}

%----------------------------------------------------------------------------------------
%	VARIOUS REQUIRED PACKAGES AND CONFIGURATIONS
%----------------------------------------------------------------------------------------
\usepackage{dashundergaps}
\usepackage{anyfontsize}
\usepackage{qrcode}

\usepackage{pgf,tikz,pgfplots}
\pgfplotsset{compat=1.15}
\usepackage{mathrsfs}
\usetikzlibrary{arrows}


\usepackage[top=3cm,bottom=3cm,left=3cm,right=3cm,headsep=10pt,a4paper]{geometry} % Page margins

\usepackage{graphicx} % Required for including pictures
\graphicspath{{Pictures/}} % Specifies the directory where pictures are stored

\usepackage{lipsum} % Inserts dummy text

\usepackage{xcolor} % Required for specifying colors by name

\definecolor{ocre}{RGB}{243,102,25} % Define the orange color used for highlighting throughout the book


\usepackage{tikz} % Required for drawing custom shapes

\usepackage[english]{babel} % English language/hyphenation

\usepackage{enumitem} % Customize lists
\setlist{nolistsep} % Reduce spacing between bullet points and numbered lists

\usepackage{booktabs} % Required for nicer horizontal rules in tables


\usepackage{xstring}
\usepackage{multido}

%----------------------------------------------------------------------------------------
%	FONTS
%----------------------------------------------------------------------------------------

\usepackage{avant} % Use the Avantgarde font for headings
%\usepackage{times} % Use the Times font for headings
\usepackage{mathptmx} % Use the Adobe Times Roman as the default text font together with math symbols from the Sym­bol, Chancery and Com­puter Modern fonts

\usepackage{microtype} % Slightly tweak font spacing for aesthetics
\usepackage[utf8]{inputenc} % Required for including letters with accents
\usepackage[T1]{fontenc} % Use 8-bit encoding that has 256 glyphs

%----------------------------------------------------------------------------------------
%	BIBLIOGRAPHY AND INDEX
%----------------------------------------------------------------------------------------

\usepackage[style=numeric,citestyle=numeric,sorting=nyt,sortcites=true,autopunct=true,babel=hyphen,hyperref=true,abbreviate=false,backref=true,backend=biber]{biblatex}
\addbibresource{bibliography.bib} % BibTeX bibliography file
\defbibheading{bibempty}{}

\usepackage{calc} % For simpler calculation - used for spacing the index letter headings correctly
\usepackage{makeidx} % Required to make an index
\makeindex % Tells LaTeX to create the files required for indexing

%----------------------------------------------------------------------------------------
%	MAIN TABLE OF CONTENTS
%----------------------------------------------------------------------------------------

\usepackage{titletoc} % Required for manipulating the table of contents

\contentsmargin{0cm} % Removes the default margin

% Part text styling
\titlecontents{part}[0cm]
{\addvspace{20pt}\centering\large\bfseries}
{}
{}
{}

% Chapter text styling
\titlecontents{chapter}[1.25cm] % Indentation
{\addvspace{12pt}\large\sffamily\bfseries} % Spacing and font options for chapters
{\color{ocre!60}\contentslabel[\Large\thecontentslabel]{1.25cm}\color{ocre}} % Chapter number
{\color{ocre}}  
{\color{ocre!60}\normalsize\;\titlerule*[.5pc]{.}\;\thecontentspage} % Page number

% Section text styling
\titlecontents{section}[1.25cm] % Indentation
{\addvspace{3pt}\sffamily\bfseries} % Spacing and font options for sections
{\contentslabel[\thecontentslabel]{1.25cm}} % Section number
{}
{\hfill\color{black}\thecontentspage} % Page number
[]

% Subsection text styling
\titlecontents{subsection}[1.25cm] % Indentation
{\addvspace{1pt}\sffamily\small} % Spacing and font options for subsections
{\contentslabel[\thecontentslabel]{1.25cm}} % Subsection number
{}
{\ \titlerule*[.5pc]{.}\;\thecontentspage} % Page number
[]

% List of figures
\titlecontents{figure}[0em]
{\addvspace{-5pt}\sffamily}
{\thecontentslabel\hspace*{1em}}
{}
{\ \titlerule*[.5pc]{.}\;\thecontentspage}
[]

% List of tables
\titlecontents{table}[0em]
{\addvspace{-5pt}\sffamily}
{\thecontentslabel\hspace*{1em}}
{}
{\ \titlerule*[.5pc]{.}\;\thecontentspage}
[]

%----------------------------------------------------------------------------------------
%	MINI TABLE OF CONTENTS IN PART HEADS
%----------------------------------------------------------------------------------------

% Chapter text styling
\titlecontents{lchapter}[0em] % Indenting
{\addvspace{15pt}\large\sffamily\bfseries} % Spacing and font options for chapters
{\color{ocre}\contentslabel[\Large\thecontentslabel]{1.25cm}\color{ocre}} % Chapter number
{}  
{\color{ocre}\normalsize\sffamily\bfseries\;\titlerule*[.5pc]{.}\;\thecontentspage} % Page number

% Section text styling
\titlecontents{lsection}[0em] % Indenting
{\sffamily\small} % Spacing and font options for sections
{\contentslabel[\thecontentslabel]{1.25cm}} % Section number
{}
{}

% Subsection text styling
\titlecontents{lsubsection}[.5em] % Indentation
{\normalfont\footnotesize\sffamily} % Font settings
{}
{}
{}

%----------------------------------------------------------------------------------------
%	THEOREM STYLES
%----------------------------------------------------------------------------------------

\usepackage{amsmath,amsfonts,amssymb,amsthm} % For math equations, theorems, symbols, etc

\newcommand{\intoo}[2]{\mathopen{]}#1\,;#2\mathclose{[}}
\newcommand{\ud}{\mathop{\mathrm{{}d}}\mathopen{}}
\newcommand{\intff}[2]{\mathopen{[}#1\,;#2\mathclose{]}}
\newtheorem{notation}{Notation}[chapter]

% Boxed/framed environments
\newtheoremstyle{ocrenumbox}% % Theorem style name
{0pt}% Space above
{0pt}% Space below
{\normalfont}% % Body font
{}% Indent amount
{\small\bf\sffamily\color{ocre}}% % Theorem head font
{\;}% Punctuation after theorem head
{0.25em}% Space after theorem head
{\small\sffamily\color{ocre}\thmname{#1}\nobreakspace\thmnumber{\@ifnotempty{#1}{}\@upn{#2}}% Theorem text (e.g. Theorem 2.1)
\thmnote{\nobreakspace\the\thm@notefont\sffamily\bfseries\color{black}---\nobreakspace#3.}} % Optional theorem note
\renewcommand{\qedsymbol}{$\blacksquare$}% Optional qed square

\newtheoremstyle{blacknumex}% Theorem style name
{5pt}% Space above
{5pt}% Space below
{\normalfont}% Body font
{} % Indent amount
{\small\bf\sffamily}% Theorem head font
{\;}% Punctuation after theorem head
{0.25em}% Space after theorem head
{\small\sffamily{\tiny\ensuremath{\blacksquare}}\nobreakspace\thmname{#1}\nobreakspace\thmnumber{\@ifnotempty{#1}{}\@upn{#2}}% Theorem text (e.g. Theorem 2.1)
\thmnote{\nobreakspace\the\thm@notefont\sffamily\bfseries---\nobreakspace#3.}}% Optional theorem note

\newtheoremstyle{blacknumbox} % Theorem style name
{0pt}% Space above
{0pt}% Space below
{\normalfont}% Body font
{}% Indent amount
{\small\bf\sffamily}% Theorem head font
{\;}% Punctuation after theorem head
{0.25em}% Space after theorem head
{\small\sffamily\thmname{#1}\nobreakspace\thmnumber{\@ifnotempty{#1}{}\@upn{#2}}% Theorem text (e.g. Theorem 2.1)
\thmnote{\nobreakspace\the\thm@notefont\sffamily\bfseries---\nobreakspace#3.}}% Optional theorem note

% Non-boxed/non-framed environments
\newtheoremstyle{ocrenum}% % Theorem style name
{5pt}% Space above
{5pt}% Space below
{\normalfont}% % Body font
{}% Indent amount
{\small\bf\sffamily\color{ocre}}% % Theorem head font
{\;}% Punctuation after theorem head
{0.25em}% Space after theorem head
{\small\sffamily\color{ocre}\thmname{#1}\nobreakspace\thmnumber{\@ifnotempty{#1}{}\@upn{#2}}% Theorem text (e.g. Theorem 2.1)
\thmnote{\nobreakspace\the\thm@notefont\sffamily\bfseries\color{black}---\nobreakspace#3.}} % Optional theorem note
\renewcommand{\qedsymbol}{$\blacksquare$}% Optional qed square
\makeatother

% Defines the theorem text style for each type of theorem to one of the three styles above
\newcounter{dummy} 
\numberwithin{dummy}{section}
\theoremstyle{ocrenumbox}
\newtheorem{theoremeT}[dummy]{Theorem}
\newtheorem{problem}{Problem}[chapter]
\newtheorem{exerciseT}{Exercise}[chapter]
\theoremstyle{blacknumex}
\newtheorem{exampleT}{Example}[chapter]
\theoremstyle{blacknumbox}
\newtheorem{vocabulary}{Vocabulary}[chapter]
\newtheorem{definitionT}{Definition}[section]
\newtheorem{corollaryT}[dummy]{Corollary}
\theoremstyle{ocrenum}
\newtheorem{proposition}[dummy]{Proposition}

%----------------------------------------------------------------------------------------
%	DEFINITION OF COLORED BOXES
%----------------------------------------------------------------------------------------

\RequirePackage[framemethod=default]{mdframed} % Required for creating the theorem, definition, exercise and corollary boxes

% Theorem box
\newmdenv[skipabove=7pt,
skipbelow=7pt,
backgroundcolor=black!5,
linecolor=ocre,
innerleftmargin=5pt,
innerrightmargin=5pt,
innertopmargin=5pt,
leftmargin=0cm,
rightmargin=0cm,
innerbottommargin=5pt]{tBox}

% Exercise box	  
\newmdenv[skipabove=7pt,
skipbelow=7pt,
rightline=false,
leftline=true,
topline=false,
bottomline=false,
backgroundcolor=ocre!10,
linecolor=ocre,
innerleftmargin=5pt,
innerrightmargin=5pt,
innertopmargin=5pt,
innerbottommargin=5pt,
leftmargin=0cm,
rightmargin=0cm,
linewidth=4pt]{eBox}	

% Definition box
\newmdenv[skipabove=7pt,
skipbelow=7pt,
rightline=false,
leftline=true,
topline=false,
bottomline=false,
linecolor=ocre,
innerleftmargin=5pt,
innerrightmargin=5pt,
innertopmargin=0pt,
leftmargin=0cm,
rightmargin=0cm,
linewidth=4pt,
innerbottommargin=0pt]{dBox}	

% Corollary box
\newmdenv[skipabove=7pt,
skipbelow=7pt,
rightline=false,
leftline=true,
topline=false,
bottomline=false,
linecolor=gray,
backgroundcolor=black!5,
innerleftmargin=5pt,
innerrightmargin=5pt,
innertopmargin=5pt,
leftmargin=0cm,
rightmargin=0cm,
linewidth=4pt,
innerbottommargin=5pt]{cBox}

% Creates an environment for each type of theorem and assigns it a theorem text style from the "Theorem Styles" section above and a colored box from above


%-------------------------------------------------------------------------------
% Note: Comment by Guo DeLiang
% Date: 15/02/2020
%
%-------------------------------------------------------------------------------
%% \newenvironment{theorem}{\begin{tBox}\begin{theoremeT}}{\end{theoremeT}\end{tBox}}

% \newtcbtheorem[auto counter,number within=section]{theo}%
  % {Theorem}{fonttitle=\bfseries\upshape, fontupper=\slshape,
     % arc=0mm, colback=blue!5!white,colframe=blue!75!black}{theorem}

\newenvironment{exercise}{\begin{eBox}\begin{exerciseT}}{\hfill{\color{ocre}\tiny\ensuremath{\blacksquare}}\end{exerciseT}\end{eBox}}				  
\newenvironment{definition}{\begin{dBox}\begin{definitionT}}{\end{definitionT}\end{dBox}}	
\newenvironment{example}{\begin{exampleT}}{\hfill{\tiny\ensuremath{\blacksquare}}\end{exampleT}}		
\newenvironment{corollary}{\begin{cBox}\begin{corollaryT}}{\end{corollaryT}\end{cBox}}	

%----------------------------------------------------------------------------------------
%	REMARK ENVIRONMENT
%----------------------------------------------------------------------------------------

\newenvironment{remark}{\par\vspace{10pt}\small % Vertical white space above the remark and smaller font size
\begin{list}{}{	
\leftmargin=35pt % Indentation on the left
\rightmargin=25pt}\item\ignorespaces % Indentation on the right
\makebox[-2.5pt]{\begin{tikzpicture}[overlay]
\node[draw=ocre!60,line width=1pt,circle,fill=ocre!25,font=\sffamily\bfseries,inner sep=2pt,outer sep=0pt] at (-15pt,0pt){\textcolor{ocre}{R}};\end{tikzpicture}} % Orange R in a circle
\advance\baselineskip -1pt}{\end{list}\vskip5pt} % Tighter line spacing and white space after remark

%----------------------------------------------------------------------------------------
%	SECTION NUMBERING IN THE MARGIN
%----------------------------------------------------------------------------------------

\makeatletter
\renewcommand{\@seccntformat}[1]{\llap{\textcolor{ocre}{\csname the#1\endcsname}\hspace{1em}}}                    
\renewcommand{\section}{\@startsection{section}{1}{\z@}
{-4ex \@plus -1ex \@minus -.4ex}
{1ex \@plus.2ex }
{\normalfont\large\sffamily\bfseries}}
\renewcommand{\subsection}{\@startsection {subsection}{2}{\z@}
{-3ex \@plus -0.1ex \@minus -.4ex}
{0.5ex \@plus.2ex }
{\normalfont\sffamily\bfseries}}
\renewcommand{\subsubsection}{\@startsection {subsubsection}{3}{\z@}
{-2ex \@plus -0.1ex \@minus -.2ex}
{.2ex \@plus.2ex }
{\normalfont\small\sffamily\bfseries}}                        
\renewcommand\paragraph{\@startsection{paragraph}{4}{\z@}
{-2ex \@plus-.2ex \@minus .2ex}
{.1ex}
{\normalfont\small\sffamily\bfseries}}

%----------------------------------------------------------------------------------------
%	PART HEADINGS
%----------------------------------------------------------------------------------------

% numbered part in the table of contents
\newcommand{\@mypartnumtocformat}[2]{%
\setlength\fboxsep{0pt}%
\noindent\colorbox{ocre!20}{\strut\parbox[c][.7cm]{\ecart}{\color{ocre!70}\Large\sffamily\bfseries\centering#1}}\hskip\esp\colorbox{ocre!40}{\strut\parbox[c][.7cm]{\linewidth-\ecart-\esp}{\Large\sffamily\centering#2}}}%
%%%%%%%%%%%%%%%%%%%%%%%%%%%%%%%%%%
% unnumbered part in the table of contents
\newcommand{\@myparttocformat}[1]{%
\setlength\fboxsep{0pt}%
\noindent\colorbox{ocre!40}{\strut\parbox[c][.7cm]{\linewidth}{\Large\sffamily\centering#1}}}%
%%%%%%%%%%%%%%%%%%%%%%%%%%%%%%%%%%
\newlength\esp
\setlength\esp{4pt}
\newlength\ecart
\setlength\ecart{1.2cm-\esp}
\newcommand{\thepartimage}{}%
\newcommand{\partimage}[1]{\renewcommand{\thepartimage}{#1}}%
\def\@part[#1]#2{%
\ifnum \c@secnumdepth >-2\relax%
\refstepcounter{part}%
\addcontentsline{toc}{part}{\texorpdfstring{\protect\@mypartnumtocformat{\thepart}{#1}}{\partname~\thepart\ ---\ #1}}
\else%
\addcontentsline{toc}{part}{\texorpdfstring{\protect\@myparttocformat{#1}}{#1}}%
\fi%
\startcontents%
\markboth{}{}%
{\thispagestyle{empty}%
\begin{tikzpicture}[remember picture,overlay]%
\node at (current page.north west){\begin{tikzpicture}[remember picture,overlay]%	
\fill[ocre!20](0cm,0cm) rectangle (\paperwidth,-\paperheight);
\node[anchor=north] at (4cm,-3.25cm){\color{ocre!40}\fontsize{220}{100}\sffamily\bfseries\thepart}; 
\node[anchor=south east] at (\paperwidth-1cm,-\paperheight+1cm){\parbox[t][][t]{8.5cm}{
\printcontents{l}{0}{\setcounter{tocdepth}{1}}%
}};
\node[anchor=north east] at (\paperwidth-1.5cm,-3.25cm){\parbox[t][][t]{15cm}{\strut\raggedleft\color{white}\fontsize{30}{30}\sffamily\bfseries#2}};
\end{tikzpicture}};
\end{tikzpicture}}%
\@endpart}
\def\@spart#1{%
\startcontents%
\phantomsection
{\thispagestyle{empty}%
\begin{tikzpicture}[remember picture,overlay]%
\node at (current page.north west){\begin{tikzpicture}[remember picture,overlay]%	
\fill[ocre!20](0cm,0cm) rectangle (\paperwidth,-\paperheight);
\node[anchor=north east] at (\paperwidth-1.5cm,-3.25cm){\parbox[t][][t]{15cm}{\strut\raggedleft\color{white}\fontsize{30}{30}\sffamily\bfseries#1}};
\end{tikzpicture}};
\end{tikzpicture}}
\addcontentsline{toc}{part}{\texorpdfstring{%
\setlength\fboxsep{0pt}%
\noindent\protect\colorbox{ocre!40}{\strut\protect\parbox[c][.7cm]{\linewidth}{\Large\sffamily\protect\centering #1\quad\mbox{}}}}{#1}}%
\@endpart}
\def\@endpart{\vfil\newpage
\if@twoside
\if@openright
\null
\thispagestyle{empty}%
\newpage
\fi
\fi
\if@tempswa
\twocolumn
\fi}

%----------------------------------------------------------------------------------------
%	CHAPTER HEADINGS
%----------------------------------------------------------------------------------------

% A switch to conditionally include a picture, implemented by  Christian Hupfer
\newif\ifusechapterimage
\usechapterimagetrue
\newcommand{\thechapterimage}{}%
\newcommand{\chapterimage}[1]{\ifusechapterimage\renewcommand{\thechapterimage}{#1}\fi}%
\newcommand{\autodot}{.}
\def\@makechapterhead#1{%
{\parindent \z@ \raggedright \normalfont
\ifnum \c@secnumdepth >\m@ne
\if@mainmatter
\begin{tikzpicture}[remember picture,overlay]
\node at (current page.north west)
{\begin{tikzpicture}[remember picture,overlay]
\node[anchor=north west,inner sep=0pt] at (0,0) {\ifusechapterimage\includegraphics[width=\paperwidth]{\thechapterimage}\fi};
\draw[anchor=west] (\Gm@lmargin,-9cm) node [line width=2pt,rounded corners=15pt,draw=ocre,fill=white,fill opacity=0.5,inner sep=15pt]{\strut\makebox[22cm]{}};
\draw[anchor=west] (\Gm@lmargin+.3cm,-9cm) node {\huge\sffamily\bfseries\color{black}\thechapter\autodot~#1\strut};
\end{tikzpicture}};
\end{tikzpicture}
\else
\begin{tikzpicture}[remember picture,overlay]
\node at (current page.north west)
{\begin{tikzpicture}[remember picture,overlay]
\node[anchor=north west,inner sep=0pt] at (0,0) {\ifusechapterimage\includegraphics[width=\paperwidth]{\thechapterimage}\fi};
\draw[anchor=west] (\Gm@lmargin,-9cm) node [line width=2pt,rounded corners=15pt,draw=ocre,fill=white,fill opacity=0.5,inner sep=15pt]{\strut\makebox[22cm]{}};
\draw[anchor=west] (\Gm@lmargin+.3cm,-9cm) node {\huge\sffamily\bfseries\color{black}#1\strut};
\end{tikzpicture}};
\end{tikzpicture}
\fi\fi\par\vspace*{270\p@}}}

%-------------------------------------------

\def\@makeschapterhead#1{%
\begin{tikzpicture}[remember picture,overlay]
\node at (current page.north west)
{\begin{tikzpicture}[remember picture,overlay]
\node[anchor=north west,inner sep=0pt] at (0,0) {\ifusechapterimage\includegraphics[width=\paperwidth]{\thechapterimage}\fi};
\draw[anchor=west] (\Gm@lmargin,-9cm) node [line width=2pt,rounded corners=15pt,draw=ocre,fill=white,fill opacity=0.5,inner sep=15pt]{\strut\makebox[22cm]{}};
\draw[anchor=west] (\Gm@lmargin+.3cm,-9cm) node {\huge\sffamily\bfseries\color{black}#1\strut};
\end{tikzpicture}};
\end{tikzpicture}
\par\vspace*{270\p@}}
\makeatother

%----------------------------------------------------------------------------------------
%	HYPERLINKS IN THE DOCUMENTS
%----------------------------------------------------------------------------------------

			
			
\usepackage{bookmark}
\bookmarksetup{
           open,
           numbered,
           addtohook={%
\ifnum\bookmarkget{level}=0 % chapter
\bookmarksetup{bold}%
\fi
\ifnum\bookmarkget{level}=-1 % part
\bookmarksetup{color=ocre,bold}%
\fi
}
}




%----------------------------------------------------------------------------------------
%	Added by Guo De Liang Thomas
%----------------------------------------------------------------------------------------
\usepackage[SC5b]{ean13isbn}
\usepackage{todonotes}
\usepackage{multicol}
\usepackage{multiaudience}

% \SetNewAudience{Tutor}
% \SetNewAudience{Parent}
% \SetNewAudience{Student}
% \SetNewAudience{Solution}
% \SetNewAudience{MarkScheme}


\usepackage{versions}
\usepackage[final]{pdfpages}
\usepackage{tasks}
\usepackage[makeroom]{cancel}

\usepackage{CJKutf8}               % Note: For Chinese Characters
%\usepackage{ctex}

\let\exercise\pkexercise




%-----------------------------------------------------------
% Setup: xsim 
%
%
%-----------------------------------------------------------

\usepackage{xsim}

\usepackage{tcolorbox}
\tcbuselibrary{breakable, skins, listings}

\tcbset{boxrule=0pt,sharp corners}


\xsimsetup{
      exercise/name =
    , exercise/template  = easyitem
    , exercise/print     = true
    , solution/template  = easyitem
    , solution/print     = true
    , blank/blank-style  = \dotuline{\textcolor{blue}{#1}}
    , blank/filled-style = \dotuline{\makebox[\textwidth][l]{\textcolor{blue}{#1}}}
}

%-----------------------------------------------------------
% Below script block is from the manua of XSIM
%
%
%-----------------------------------------------------------

\DeclareExerciseEnvironmentTemplate{easyitem}
  {\item[\GetExerciseProperty{counter}]}
  {}


\DeclareExerciseEnvironmentTemplate{boxed}
  {%
    \tcolorbox[
      enhanced ,
      % attach boxed title to top center = {yshift=-7.5pt} ,
      colback = white , 
	  colbacktitle = white ,
      coltitle = black , 
	  % colframe = black ,
      boxed title style = { colframe = black } ,
      fonttitle=\bfseries,
      sharp corners=all,
      breakable,
      title=
        \XSIMmixedcase{\GetExerciseName} % -- Comment the exercise name
		\textbf{\GetExerciseName ~\GetExerciseProperty{counter}}
        \IfInsideSolutionF{%
          ~\GetExerciseProperty{counter}%
           \IfExercisePropertySetT{subtitle}{. \GetExerciseProperty{subtitle}}%
        }%
    ]
	
    \IfInsideSolutionT{%
     % \tcbsubtitle{Exercise}
     %     \GetExerciseBody{exercise}  %% Include exercise body
          \tcbsubtitle{Solution}
    }
	
  }
  {
    \Pointilles[1.5]{4}
    \endtcolorbox
  }



% declare a user command for short answers:
\NewDocumentCommand\answer{m}{%
  \IfSolutionPrintT{%
    \UseExerciseTemplate{begin}{solution}%
      #1%
    \UseExerciseTemplate{end}{solution}%
  }{}%
}

\usepackage{intcalc}
\setcounter{tocdepth}{5}





%-------------------------------------------------------------------------------------------------
% Theorem
%        
% https://texblog.org/2015/09/30/fancy-boxes-for-theorem-lemma-and-proof-with-mdframed/
%
%-------------------------------------------------------------------------------------------------

% \newcounter{theo}[section] \setcounter{theo}{0}
% \renewcommand{\thetheo}{\arabic{section}.\arabic{theo}}
% \newenvironment{theo}[2][]{%
% \refstepcounter{theo}%
% \ifstrempty{#1}%
% {\mdfsetup{%
% frametitle={%
% \tikz[baseline=(current bounding box.east),outer sep=0pt]
% \node[anchor=east,rectangle,fill=blue!20]
% {\strut Theorem~\thetheo};}}
% }%
% {\mdfsetup{%
% frametitle={%
% \tikz[baseline=(current bounding box.east),outer sep=0pt]
% \node[anchor=east,rectangle,fill=blue!20]
% {\strut Theorem~\thetheo:~#1};}}%
% }%
% \mdfsetup{innertopmargin=10pt,linecolor=blue!20,%
% linewidth=2pt,topline=true,%
% frametitleaboveskip=\dimexpr-\ht\strutbox\relax
% }
% \begin{mdframed}[]\relax%
% \label{#2}}{\end{mdframed}}



%-------------------------------------------------------------------------------------------------
% Lemma
%        
% https://texblog.org/2015/09/30/fancy-boxes-for-theorem-lemma-and-proof-with-mdframed/
%
%-------------------------------------------------------------------------------------------------

\newcounter{lem}[section] \setcounter{lem}{0}
\renewcommand{\thelem}{\arabic{section}.\arabic{lem}}
\newenvironment{lem}[2][]{%
\refstepcounter{lem}%
\ifstrempty{#1}%
{\mdfsetup{%
    frametitle={%
    \tikz[baseline=(current bounding box.east),outer sep=0pt]
    \node[anchor=east,rectangle,fill=green!20]
    {\strut Lemma~\thelem};}}
}%
{\mdfsetup{%
    frametitle={%
    \tikz[baseline=(current bounding box.east),outer sep=0pt]
    \node[anchor=east,rectangle,fill=green!20]
    {\strut Lemma~\thelem:~#1};}}%
}%
\mdfsetup{innertopmargin=10pt,linecolor=green!20,%
    linewidth=2pt,topline=true,%
    frametitleaboveskip=\dimexpr-\ht\strutbox\relax
}
\begin{mdframed}[]\relax%
    \label{#2}}{\end{mdframed}}



%-------------------------------------------------------------------------------------------------
% Proof
%        
% https://texblog.org/2015/09/30/fancy-boxes-for-theorem-lemma-and-proof-with-mdframed/
%
%-------------------------------------------------------------------------------------------------
\newcounter{prf}[section]\setcounter{prf}{0}
\renewcommand{\theprf}{\arabic{section}.\arabic{prf}}
\newenvironment{prf}[2][]{%
\refstepcounter{prf}%
\ifstrempty{#1}%
{\mdfsetup{%
frametitle={%
\tikz[baseline=(current bounding box.east),outer sep=0pt]
\node[anchor=east,rectangle,fill=red!20]
{\strut Proof~\theprf};}}
}%
{\mdfsetup{%
frametitle={%
\tikz[baseline=(current bounding box.east),outer sep=0pt]
\node[anchor=east,rectangle,fill=red!20]
{\strut Proof~\theprf:~#1};}}%
}%
\mdfsetup{innertopmargin=10pt,linecolor=red!20,%
linewidth=2pt,topline=true,%
frametitleaboveskip=\dimexpr-\ht\strutbox\relax
}
\begin{mdframed}[]\relax%
\label{#2}}{\qed\end{mdframed}}
%%%%%%%%%%%%%%%%%%%%%%%%%%%%%%







%-------------------------------------------------------------------------------------------------
% Skill
%        
% https://texblog.org/2015/09/30/fancy-boxes-for-theorem-lemma-and-proof-with-mdframed/
%
%-------------------------------------------------------------------------------------------------

\newcounter{skillList}[section] \setcounter{skillList}{0}
\renewcommand{\theskillList}{\arabic{skillList}}
\newenvironment{skillList}[2][]{%
\refstepcounter{skillList}%
\ifstrempty{#1}%
{\mdfsetup{%
frametitle={%
\tikz[baseline=(current bounding box.east),outer sep=0pt]
\node[anchor=east,rectangle,fill=blue!20]
{\strut Skill~\theskillList};}}
}%
{\mdfsetup{%
frametitle={%
\tikz[baseline=(current bounding box.east),outer sep=0pt]
\node[anchor=east,rectangle,fill=blue!20]
{\strut Skill~\theskillList:~#1};}}%
}%
\mdfsetup{innertopmargin=10pt,linecolor=blue!20,%
linewidth=2pt,topline=true,%
frametitleaboveskip=\dimexpr-\ht\strutbox\relax
}
\begin{mdframed}[]\relax%
\label{#2}}{\end{mdframed}}



%-------------------------------------------------------------------------------------------------





%-------------------------------------------------------------------------------------------------------------
%	https://tex.stackexchange.com/questions/28087/example-of-fancy-table-using-tikz-package
%
%
%
%
%-------------------------------------------------------------------------------------------------------------
\usetikzlibrary{matrix}

\tikzset{ 
    table/.style={
        matrix of nodes,
        row sep=-\pgflinewidth,
        column sep=-\pgflinewidth,
        nodes={
            rectangle,
            draw=black,
            align=center
        },
        minimum height=1.5em,
        text depth=0.5ex,
        text height=2ex,
        nodes in empty cells,
%%
        every odd row/.style={
            nodes={fill=gray!20}
        },
        column 1/.style={
            nodes={text width=2em,font=\bfseries}
        },
        row 1/.style={
            nodes={
                fill=gray!40,
                text=black,
                font=\bfseries
            }
        }
    }
}



%---------------------------------------------------------------------------------------------
% Description:
%     Set the type of final output
%
% Sample:
%     biBook       - Ebook
%
%     biPastpaper  - Due to copyright issue, need to set header, trailer
%                    
%                    
%
%     biTopical    - 
%     
%
% Date: 2019.08.15
%
%     bi           - Boolean Indicator
%
%--------------------------------------------------------------------------------------------

\newtoggle{biPublish}              % Published Book
                                   %     - All materials should be legal and copyright should 
                                   %       be verified.								   

\togglefalse{biPublish}


\newtoggle{biTutorCN}              % Tutition in Chinese
\toggletrue{biTutorCN}

\newtoggle{biPrintOnly}            % bi Boolean Indicator - Print Only
\newtoggle{biNoCoverPage}          % bi Boolean Indicator - No Print Cover Page

\newtoggle{biStudent}              % bi Boolean Indicator - Student

\newtoggle{biPastPaper}            % bi Boolean Indicator - Past Paper
\newtoggle{biTopicTest}            % bi Boolean Indicator - Topic Test
\newtoggle{biWorksheet}            % bi Boolean Indicator - Worksheet
\newtoggle{biExcercise}            % bi Boolean Indicator - Excercise


\newtoggle{biSolution}             % bi Boolean Indicator - Solution
\newtoggle{biMarkScheme}           % bi Boolean Indicator - Mark Scheme



\newtoggle{biTutor}                % bi Boolean Indicator - Tutor


\newtoggle{biChinese}              % bi Boolean Indicator - Include Chinese.
\toggletrue{biChinese}     


%-----------------------------------------------------------
% Option Setup:
%     1. bTextbook
%     2. bRevision
%     3. bWorkbook
%     4. bTest
%     5. bPastPaper
%
%     6. bQuestion
%     7. bAnswer
%-----------------------------------------------------------

\newtoggle{bTextbook}
\newtoggle{bRevision}
\newtoggle{bWorkbook}
\newtoggle{bTest}
\newtoggle{bPastPaper}

\newtoggle{bQuestion}
\newtoggle{bSolution}
\newtoggle{bAnswer}

\newtoggle{bDottedLine}

\newtoggle{biMacleans}             % bi Boolean Indicator
\newtoggle{biStKent}
\newtoggle{biACGStrathAllan} 

\newcounter{nOddEven}

\newcounter{nExamYear}

\newcounter{nLevelStart}
\setcounter{nLevelStart}{1}    %------ Level

\newcounter{nLevelEnd}
\setcounter{nLevelEnd}{1800}   %------ Level


\newcounter{nWorkBook}



%-----------------------------------------------------------------------------------------------------
% Low Level Parameter Setup
%
%
%
%
%-----------------------------------------------------------------------------------------------------

\togglefalse{bTextbook}
\toggletrue{bRevision}
\togglefalse{bWorkbook}
\toggletrue{bTest}
\togglefalse{bPastPaper}

\toggletrue{bDottedLine}

%\toggletrue{biPrintOnly}
\toggletrue{biNoCoverPage}

\iftoggle{biNoCoverPage}{ \renewcommand{\iPageStart}{2} }
                        { \renewcommand{\iPageStart}{1} }

\toggletrue{biTutor}
\toggletrue{biStudent}
%\toggletrue{biBook}
%\toggletrue{biSolution}

\togglefalse{biMacleans}
%\toggletrue{biStKent}
%\toggletrue{biACGStrathAllan} 


\toggletrue{bQuestion}
\togglefalse{bAnswer}

\setcounter{nExamYear}{00}     %------ Exam Year 2017


\setcounter{nWorkBook}{2}      %------ 0 Workbook A - Year 2017, 2015 
                               %------ 1 Workbook B - Year 2016, 2014
                               %------ 2 Workbook   - Year 2017, 2016, 2015, 2014....
							 



%-----------------------------------------------------------------------------------------------------
% High Level Parameter Setup
%
%
%
%
%-----------------------------------------------------------------------------------------------------






%----------------------------------------------------------------------------------------
%	PAGE HEADERS
%----------------------------------------------------------------------------------------

\usepackage{fancyhdr} % Required for header and footer configuration

\pagestyle{fancy}
\renewcommand{\chaptermark}[1]{\markboth{\sffamily\normalsize\bfseries\chaptername\ \thechapter.\ #1}{}} % Chapter text font settings
\renewcommand{\sectionmark}[1]{\markright{\sffamily\normalsize\thesection\hspace{5pt}#1}{}} % Section text font settings
\fancyhf{} 
\fancyhead[LE,RO]{\sffamily\normalsize\thepage} % Font setting for the page number in the header
\fancyhead[LO]{\rightmark} % Print the nearest section name on the left side of odd pages
%\fancyhead[CE,CO]{Date: \underline{\hspace{2cm}} Time: \underline{\hspace{1cm}}}
\fancyhead[RE]{\leftmark}  % Print the current chapter name on the right side of even pages
\renewcommand{\headrulewidth}{0.5pt} % Width of the rule under the header
\addtolength{\headheight}{2.5pt} % Increase the spacing around the header slightly
\renewcommand{\footrulewidth}{0pt} % Removes the rule in the footer


%\fancyfoot[LE,RO]{Guo De Liang Thomas}
%\fancyfoot[LO]{\textsuperscript{\textcopyright}2018 Cockle Bay Elite Education} % Print the nearest section name on the left side of odd pages
%\fancyfoot[RE]{\textsuperscript{\textcopyright}2018 Cockle Bay Elite Education} % Print the current chapter name on the right side of even pages

% \renewcommand{\footrulewidth}{2.5pt} % Removes the rule in the footer

\renewcommand{\footrulewidth}{0.4pt}

\iftoggle{biPublish}{
       \fancyfoot[L]{\href{http://www.eliteducation.net}{\textsuperscript{\textcopyright}Elite Education 2019}}
	   \fancyfoot[R]{\href{http://www.eliteducation.net}{http://www.eliteducation.net}}
	   
       % \fancyfoot[LE,RO]{\href{http://www.eliteducation.net}{\textsuperscript{\textcopyright}Elite Education 2019 http://www.eliteducation.net}}
	}
	{
       \fancyfoot[LE,RO]{KiwiEdu9@gmail.com}  
	}


\iftoggle{biTutorCN}{
       \fancyfoot[L]{\href{http://www.eliteducation.net}{\textsuperscript{\textcopyright} \begin{CJK*}{UTF8}{gbsn}精英教育 私塾辅导\end{CJK*}   }}
	   \fancyfoot[R]{\href{http://www.eliteducation.net}{  \begin{CJK*}{UTF8}{gbsn}资料整理:郭德良\end{CJK*}      }}
	   
       % \fancyfoot[LE,RO]{\href{http://www.eliteducation.net}{\textsuperscript{\textcopyright}Elite Education 2019 http://www.eliteducation.net}}
	}
	{
       \fancyfoot[LE,RO]{KiwiEdu9@gmail.com  }  
	}



\fancypagestyle{plain}{\fancyhead{}\renewcommand{\headrulewidth}{0pt}} % Style for when a plain pagestyle is specified

% Removes the header from odd empty pages at the end of chapters
\makeatletter
\renewcommand{\cleardoublepage}{
\clearpage\ifodd\c@page\else
\hbox{}
\vspace*{\fill}
\thispagestyle{empty}
\newpage
\fi}

							 
\setcounter{secnumdepth}{5}
\setcounter{tocdepth}{4}



%------------------------------------------------------------------------------------
% Description: 
%     To display or hide certain informations based on the following tags:
%
%      tgRefBook    - Referenced book information
%      tgRefExamQ   - Referenced past paper question
%      tgRefWeb     - Referenced web URL and information
%  
%      tgExamQCIE   - To display past papers of Cambridge
%      tgExamQNCEA  - To display past papers of NCEA     
%
%      tgShowSolution - To display solution
%
%
%                   - To prepare different version of documents such as textbooks, workbooks
%        
%        
%
%
% URL:
%  
%
% Date(yyyy.mm.dd): 
%     2020.04.04
%
%--------------------------------------------------------------------
\usepackage{tagging}




\usetag{tgRefBook}
\usetag{tgRefExamQ}
\usetag{tgRefWeb} 

\usetag{tgShowSolution}

\newtcbox{\bxtgRefBook}{colframe=black}

%--------------------------------------------------------------------
% Description:
%     Create multiple dotted lines
%
% URL:
%     https://tex.stackexchange.com/questions/248451/how-to-create-multiple-dotted-lines
%
% Date: 2019.07.07
%
%--------------------------------------------------------------------


\newcommand{\Pointilles}[2][3]{%
    \iftoggle{bDottedLine}{
        \par\nobreak
        \noindent\rule{0pt}{1.5\baselineskip}% Provides a larger gap between the preceding paragraph and the dots
        \multido{}{#2}{\noindent\makebox[\linewidth]{\rule{0pt}{#1\baselineskip}\dotfill}\endgraf}% ... dotted lines ...
        \bigskip% Gap between dots and next paragraph
    }{}  
}





%--------------------------------------------------------------------
% Description:
%     Add page break before each section
%
% Sample:
%     \Pointilles[1.5]{3}
%
% URL:
%     https://tex.stackexchange.com/questions/131015/page-break-after-every-section
%
% Date: 2019.06.26
%
%--------------------------------------------------------------------

\let\oldsection\section
\renewcommand\section{\clearpage\oldsection}



\let\oldsubsection\subsection
\renewcommand\subsection{\clearpage\oldsubsection}

\let\oldsubsubsection\subsubsection
\renewcommand\subsubsection{\clearpage\oldsubsubsection}





%-------------------------------------------------------------------------------------------------


%-------------------------------------------------------------------------------------------------



%-------------------------------------------------------------------------------------------------

\providecommand{\cmdIGCSE}{}
\renewcommand{\cmdIGCSE}[9]
{
    %---------------------------------------------------------------------------------------------
    % Exam Questoins by Topic
    % 
    % For example:
    %            0580   - Exam Code  
    %	         17     - Year 2017
    %            s      - Summer Paper  	
    %            23     - Paper Variance
    %            1      - Question      (For future use)     
    %            5-7    - Pages in Exam
    %            1101   - Difficulty Level         
    %            1      - Code: 
    %                        1. Question
    %                        2. Sample Answer
    %                        3. Mark Scheme
	%            1000   - Skill Code
    %---------------------------------------------------------------------------------------------
    
    \ifnumcomp{1}{>}{1}{#5}{     % -- For future use
    }
	
    \ifnumcomp{#2}{>}{\value{nExamYear}}
    {
        \ifnumcomp{#7}{>}{\value{nLevelStart}}
        {
	        \ifnumcomp{#7}{<}{\value{nLevelEnd}}
	        {
                \setcounter{nOddEven}{\intcalcMod{#2}{2}}
                \ifnumequal{\value{nWorkBook}}{\value{nOddEven}}{}{ %--- Not Equal
		             \ifnumequal{#8}{1}{
		                \ifboolexpr{togl{biTutor} or togl{biPrintOnly} or togl{biStudent} } { 
				           \includepdf[scale=0.95
						               , pages=#6
									   , pagecommand={\pagestyle{fancy}}
									   ]
		                    {../../Education/ExamPaper/Cambridge_Ages_14_16_IGCSE/Mathematics_0580/0580_#3#2_qp_#4.pdf}
							
			            }
	                 }{}
		        }
	        }{}
	   }{}		
	}{}
}






%---------------------------------------------------------------------------------------------
% Exam Questoins by Topic
%
%      \cmdQuestionCIE{9709}{18}{s}{31}{10i}{5-7}{1110}{1}{Vector, Distance, Point, Line}
%  	
% For example:
%            9709   - Exam Code
%	         s17    - Year 2017 Summer Paper         	
%            23     - Paper Variance
%            10i    - Question (For future use)     
%            5-7    - Pages in Exam
%            1101   - Difficulty Level         
%            1      - Code: 
%                        1. Question
%                        2. Worked Solution
%                        3. Mark Scheme
%                         
%						 4. Print All
%            1000   - Skill Code
%            Surd   - Skill Description
%
% Date:
%
%---------------------------------------------------------------------------------------------

\providecommand{\cmdQuestionCIE}{}
\renewcommand{\cmdQuestionCIE}[9]
{

	\ifnumequal{#8}{1}{
	    \ifboolexpr{togl{biTutor} or togl{biPrintOnly} or togl{biStudent} } { 
	             \includepdf[pages=#6
 			            % , scale = 0.8
		                      % , clip=10mm 10mm 10mm 10mm
                              % , trim=30mm 30mm 20mm 20mm
			                  % , pagecommand={\pagestyle{fancy}}
					  		  % , frame
							]
					        	 {../../Education/ExamPaper/Cambridge_Ages_16_19_Cambridge_Advanced/A_Level/Mathematics_9709/9709_#3#2_qp_#4.pdf}
		}
    }{}
}





%---------------------------------------------------------------------------------------------
% Get the Cambridge Past Paper. 
%
%
%      \cmdPaperCIE{9709}{m19}{13}
%      \cmdPaperCIE{9709}{w19}{73}
%   	
% For example:
%            9709   - Exam Code
%	         s17    - Year 2017 Summer Paper         	
%            23     - Paper Variance
%            1      - Question      (For future use)     
%            5-7    - Pages in Exam
%            1101   - Difficulty Level         
%            1      - Code: 
%                        1. Question
%                        2. Sample Answer
%                        3. Mark Scheme
%            1000   - Skill Code
%            Surd   - Skill Description
%
% Date:
%
%---------------------------------------------------------------------------------------------

\providecommand{\cmdPaperCIE}{}
\renewcommand{\cmdPaperCIE}[3]
{
    \ifboolexpr{togl{biTutor} or togl{biPrintOnly} or togl{biStudent} } {
	    \IfFileExists{../../Education/ExamPaper/Cambridge/#1_#2_qp_#3.pdf}{    
            \includepdf[pages=\iPageStart- ,  
		               addtotoc={\iPageStart, subsection, 1, #1\_#2\_qp\_#3, lable}]
                 {../../Education/ExamPaper/Cambridge/#1_#2_qp_#3.pdf}
		}{}
	}

    \ifboolexpr{togl{biTutor} or togl{biStudent} } { 
        \IfFileExists{../../Education/ExamPaper/Cambridge/#1_#2_qp_#3_Solution.pdf}{
            \includepdf[pages=1- , frame, scale=0.9, pagecommand={},
		                addtotoc={1, subsubsection, 1, Worked Solution, lable}]
                {../../Education/ExamPaper/Cambridge/#1_#2_qp_#3_Solution.pdf}
        }{}
	}	

    \ifboolexpr{togl{biTutor} or togl{biStudent} } { 
        \IfFileExists{../../Education/ExamPaper/Cambridge/#1_#2_ms_#3.pdf}{
            \includepdf[pages=1- , frame, scale=0.9, pagecommand={},
		                addtotoc={1, subsubsection, 1, Mark Scheme, lable}]
                {../../Education/ExamPaper/Cambridge/#1_#2_ms_#3.pdf}
        }{}
	}	

}


%---------------------------------------------------------------------------------------------
% Get the IB Past Paper. 
%
%
%      \cmdPaperIB{IB-Math-HL-Paper1}{2017}{May}{TZ0}{}{} 
%     
%   	
% For example:
%            MATHL  - Exam Code
%	         HP1    - Year 2017 Summer Paper
%            Year   -
%            Season -         	
%            TZ     - Time Zone
%            1      - Question      (For future use)                
%            5-7    - Pages in Exam
%
%
%      Below pending to implement:   
%            1101   - Difficulty Level         
%            1      - Code: 
%                        1. Question
%                        2. Sample Answer
%                        3. Mark Scheme
%                        0. Get Question, Solution, and Mark Scheme
%
%            1000   - Skill Code
%            Surd   - Skill Description
%
% Date: 
%     2020.04.14
%     
%     
%---------------------------------------------------------------------------------------------


\providecommand{\cmdPaperIB}{}
\renewcommand{\cmdPaperIB}[6]
{

    % \IfSubStr{#6}{-}{  
	    % % Includes a - in the paper type
		
		% \StrBefore{#6}{-}[\StrPaper]
		% \StrBehind{#6}{-}[\StrOption]
		
		% The paper is \\
		  % \StrPaper  \\
		
		% The option is \\
              % \StrOption		
	% }{}
	
	\StrSubstitute{#1}{-}{_}[\dStrPaper]
	
	
    \ifboolexpr{togl{biTutor} or togl{biPrintOnly} or togl{biStudent} } { 
        \IfFileExists{../../Education/ExamPaper/IB/\dStrPaper_#2_#3_#4.pdf}{
            \includepdf[pages=\iPageStart- ,  
		               addtotoc={\iPageStart, subsection, 1, #3 #2  #4  , lable}]
                     {../../Education/ExamPaper/IB/\dStrPaper_#2_#3_#4.pdf}
		}{}
	}

    \ifboolexpr{togl{biTutor} or togl{biStudent} } {     
        \IfFileExists{../../Education/ExamPaper/IB/\dStrPaper_#2_#3_#4_Solution.pdf}{
            \includepdf[pages=1- , 
			            frame, 
			            % scale=0.9, 
			            pagecommand={},
		                addtotoc={1, subsubsection, 1, Worked Solution, lable}]
                     {../../Education/ExamPaper/IB/\dStrPaper_#2_#3_#4_Solution.pdf}
        }{}
	}	

    \ifboolexpr{togl{biTutor} or togl{biStudent} } { 
        \IfFileExists{../../Education/ExamPaper/IB/\dStrPaper_#2_#3_#4_ms.pdf}{
            \includepdf[pages=1- , 
			            frame, 
						% scale=0.9, 
						pagecommand={},
		                addtotoc={1, subsubsection, 1, Mark Scheme, lable}]
                     {../../Education/ExamPaper/IB/\dStrPaper_#2_#3_#4_ms.pdf}
        }{}	
	}

}


%--------------------------------------------------------------------
%
% URL:
%     https://en.wikibooks.org/wiki/LaTeX/Mathematics#Sums_and_integrals
%
% Date: 2020.04.04
%
%--------------------------------------------------------------------

\newcommand{\dd}{\mathop{}\,\mathrm{d}}



\dashundergapssetup{
    ,gap-number-format = \,\textsuperscript{\normalfont
                         (\thegapnumber)}
    ,gap-font = \itshape
	,gap-numbers = false
    ,teacher-gap-format = dot
	% ,teacher-mode=true 
    ,gap-widen=true
}



%-----------------------------------------------------------------------------------------------------------------------------------------
%
% URL:
%     https://tex.stackexchange.com/questions/332122/dotted-line-with-appropriate-length-for-answering-a-question
%
%
% Function:
%     1. Add dotted line with appropriate length for answering a question.
%     2. If it is not given the dotted line stretches as long as it can (to the end of line).
%
% Sample:
%     \answerline[3cm]
%     \answerline\newline
%
% Date: 2020.04.29
%
%-----------------------------------------------------------------------------------------------------------------------------------------

\makeatletter
\newcommand\answerline{\@ifnextchar[%]
    \answerlinetowidth\answerlinetoeol 
    \bigskip  % Gap between dots and next paragraph
}
\newcommand\answerlinetowidth[1][0pt]
{
    \hfill Answer: \hbox to #1{\leaders\hbox to \answerdotsep{\hss.\hss}\hfill}
}
\newcommand\answerlinetoeol{\leaders\hbox to \answerdotsep{\hss.\hss}\hfill\strut}
\newcommand\answerdotsep{6pt}
\makeatother




%-----------------------------------------------------------------------------------------------------------------------------------------
%
% URL:
%     https://www.overleaf.com/learn/latex/Environments
%
%
% Function:
%     
%     
%
% Sample:
%     
%   
%
% Date: 2020.04.30
%
%-----------------------------------------------------------------------------------------------------------------------------------------

\newenvironment{envAnswer}[2][3]
{
    \Pointilles[1.5]{#1}		   
    \hfill Answer:
	\iftoggle{bAnswer}
	{ #2
	  % \bigskip  % Gap between dots and next paragraph
	  
    }
	{
	    \answerline
	}
}
{
}


\everymath{\displaystyle}

